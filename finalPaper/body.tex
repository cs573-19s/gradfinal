% Introduction section is automatically added

Natural disasters have always been a big problem for people all across the world. Natural disasters cause many deaths as well as a lot of damages to the areas that they effect. When considering hurricanes, Hurricane Katrina 2005 was one of the worst hurricanes that the Atlantic has seen with more than 1000 deaths and 108 billion in damages \cite{blake2011deadliest}. There have been many hurricanes other than Hurricane Katrina with large amounts of deaths and damages which is an issue that is constantly being faced. This brings in the question of why are there so many deaths and damages? Is it that people are not warned well enough to evacuate? Are people not prepared enough for hurricanes? In a study performed by Ruginski et al., it was determined that the hurricane cone of uncertainty may not be the best way to show hurricane predicted trajectory paths. In the experiment many people confused the cone with area of effect and thought the line was the definite path of the hurricane \cite{ruginski2016non}. Due to this misunderstanding, many people often misinterpret hurricanes and even underestimate the amount of impact it will have a couple days in advance. 

This paper aims to get people to understand that hurricanes are unpredictable and that people should not waste time in evacuating when told to do so and to prepare ahead of time. This paper will talk a little bit about the background of hurricane related information as well as the methodology of how an animation was created in order to tell the story about hurricanes. Finally, the results found will be stated and discussed along with a final conclusion made about the project.

\section{Background}

Hurricane forecasting plays a large role in helping people understand the path of the hurricane in addition to when people should evacuate. This is a difficult job however, since hurricanes can be unpredictable. Landsea et al. performed an analysis on uncertainty with hurricanes. It was found that there was still a large amount of error when it comes to predicting the path of a hurricane and the error only increase the more time it takes for the hurricane to reach land. When a hurricane is 72 hours away from land, there is much more error than when it is 24 hours away however, even 24 hours away still had a decent amount of error \cite{landsea2013atlantic}. From this alone, it should be clear that people should not underestimate the unpredictability of hurricanes. While the cone of uncertainty does a good job showing hurricane uncertainty, many people still misunderstand what it actually means and may not evacuate when told to do so because they think they won't get hit by the hurricane as hard. In the study performed by Ruginski et al,. it was decided that an ensemble visualization was better for showing uncertainty of hurricanes because people would be able to see all the different paths that the hurricane could potentially take \cite{ruginski2016non}.

While people are warned about hurricanes and told to evacuate a couple days ahead of landfall, sometimes evacuating doesn't always go as planned. Closer to the year 2000, people started to pay more attention to evacuation plans and how they would evacuate a whole city in the span of two days. Urbina et al. performed a study on evacuation plans and found that evacuation plans had greatly improved since they were first implemented. However, this did not mean that it was great. There were still many issues surrounding traffic and large cities. Many of the larger cities like Houston and New Orleans still had major problems evacuating everyone in two days because there were not enough routes leading out of the city considering the population size \cite{urbina2003national}. This goes to show that there is still room for improvement in keeping people safe from hurricanes. 

People can also make other preparations for hurricanes in case evacuation plans don't go as planned or a hurricane takes a turn for the worse and they still need protection. It has been researched that an individual can build a hurricane safe room for about \$3100 according to Yazdani et al. This room would be able to house a family of 4 in addition to being able to resist winds up to 225 km/h and large debris flying around \cite{yazdani2005hurricane}. This information can be good in helping determine  a rough estimate of how many potential hurricane safe rooms that could have been built given an amount of damages. Additionally, there are many other ways to make hurricane proof houses. While the amount to make it is not specified, Bathon et al. present a wood-concrete-composite building that is cost efficient and would help protect homes and people from hurricanes \cite{bathon2006hurricane}. While this will keep people safe, people should still consider ways to protect their homes from floods. Even though a lot of damages are produced during the hurricane, a lot of additional damages are produced from floods. According to Koks et al., it is suggested that subsidy and insurance voucher programs be used in order to help combat affordability in flood insurance policies \cite{koks2015combining}. This goes to show that there are many ways to prepare for before and after hurricane damages.

\section{Method}

In order to help portrait people's lack of awareness with hurricanes and not preparing properly for hurricanes, a story animation will help viewers see the impacts of not listening to evacuations and not preparing for hurricanes before they hit. Robertson et al. discovered that an animation a better tool to use when trying to present information and tell a clean story whereas multiple views and other such graphs are better for analysis \cite{robertson2008effectiveness}. Since there is a clear story that is trying to be portrayed, an animation was the best fit for this kind of project. Even though an animation was used for telling the story, multiple views will still be used for some basic analysis. In order to do so, hurricane paths, hurricane casualties, hurricane damages and hurricane evacuation types were needed. 

Hurricane paths were needed in order to show where the hurricanes are hitting in the east coast. This allows the view to visualize where the hurricanes are, where they form and their path they can take. Additionally, the hurricane strength was portrayed using colors scaling from white to dark red. It was determined that using less than five different colors gave the best response time in identifying features as found by Healey. Additionally, Healey created a similarity table for colors to help choose colors with the most differentiation \cite{healey1996choosing}. Since there is a lot of blue and green on the maps, according to the similarity table, red was the color furthest away from blue and green, thus allowing users to identify the paths more easily on the map. The path of the hurricane would change color based on the category of the hurricane. This would help people see how hurricanes can progress over time as well as how strong some hurricanes are when they hit land. All of this information was gathered from the International Best Track Archive for Climate Stewardship (IBTrACS). They provide a data set with hurricane locations at different points in time all across the world as well as the wind speeds and other information. 

Hurricane casualties and damages were needed in order to help paint the picture of how dangerous some of the hurricanes have been. This data was obtained from Weather Underground which is a website containing hurricane information from 1851 to 2015. Since no data set was available containing the deaths and damages of each hurricane, hurricanes deaths and casualties between 1893 to 2012 were gathered manually and stored as a separate data set. Additionally, only hurricanes with significant amounts of deaths or damages were included in this data set. For the purpose of this project, significant amount for deaths was over 100 deaths and a significant amount of damages was over \$100 million. Between 1893 and 2012, about 100 significant hurricanes were identified and used for the animation and analysis.

Lastly, hurricane evacuation types were needed in order to see if there was a reason so many people die in hurricanes or why there are so much damages from hurricanes. There was no data set readily available for this type of information and there was no clear answers online for evacuation types for every single hurricane. Thus, the top five hurricanes with the most deaths and the top five hurricanes with the most damages were used for analyzing hurricane evacuation types. The evacuation type for each of these hurricanes was manually found online through multiple sources. If the hurricane did not have any indication of a mandatory evacuation but still had evacuations, it was considered a voluntary evacuation. 

With  all this information, an animation was used in order to create the story being told. In the animation, hurricane were plotted on a map one at a time showing where they start and animating a line to where it ends. While this is happening, the date and year is shown to give the views a sense of how long the hurricanes lasted while traveling. Whenever the animation gets to one of the significant hurricanes as stated earlier, the animation speed for plotting the hurricane is reduced significantly so that the user has time to process the information of that hurricane a little more than others. Additional information was also provided for each of these hurricanes including the name of the hurricane, amount of deaths and damages, and the evacuation type.

Once a hurricane is finished plotting, a death counter and damages counter is incremented by that hurricane's data. Once all the hurricanes are finished plotting, the values shown by these counters ends up being the total number of deaths and the total number of damages produced by all the hurricanes. Additionally, whenever these counters are updated, three bar charts are updates as seen in Figure 2. The first is a bar chart representing hurricane deaths, the second represents hurricane damages and the final one represents the number of hurricane proof rooms that could be made using the hurricane damages amount. The main point of the first two bar charts is to give the user an idea of how many deaths or damages a single hurricane might have compared to others. The last bar graph is to let people know that preparations could be made before hand to have potentially minimized the amount of damages.

\bars

Due to all the information being processed in the animations, it was decided that multiple views should be used in order to give users more specific information on each hurricane individually. As found out by Ward, data should not be represented with too many dimensions since the number of dimensions would be confusing and use up more screen space \cite{ward1994xmdvtool}. Additionally, as stated by DiBiase et al., animated maps should be used along side static maps or graphs in order to get the full value out of the animation \cite{dibiase1992animation}.  Line graphs displaying the paths of each individual hurricane was produced below the animation. In this area, the user can filter hurricanes by their hurricane category, the number of deaths, and the amount of damages. This filtering functionality would allow users to analyze the hurricanes in a multitude of ways. 

Based on the findings from Segel et al., they created a chart of what makes for a good story visualization. In this chart, one of the main features that most of the stories had were interaction between the user and the animation \cite{segel2010narrative}. As such, the animation created added an interactive element to it where the user can select an area on the map using the D3 brush tool as shown in Figure 3. Selecting this area would update all the individual line graphs below the animation to only show the paths of hurricanes in the selected area as seen in Figure 4. 

\brush
\multipleViews

Finally, an analysis is done on the animation to see any interesting anomalies in the data collected. Specifically, the change in number of deaths and the amount of damages is compared over the years to see when most of the deaths and damages are occurring. Additionally, taking a look at which type of evacuation causes the most amount of deaths and damages were analyzed. Last, the hurricane paths were analyzed to see if there are any patterns in their paths to predict any information about hurricanes. 

\section{Results}
Describe your results here. Focus on the facts and statistics. Do not add your interpretation of the results here.

From our analysis, there were quite a few interesting points that were found. First, looking at number of deaths, it was clear that there were significantly more deaths before the year 2000 compared to after 2000. The opposite can be said about the amount of damages where most of the damages were happening after the year 2000 compared to before 2000. Table 1 shows that there were 4 times as many damages after the year 200 than before 2000 and deaths were nearly reduced by a factor of 12 which is interesting to note.

\deathsDamagesTable

Taking a look at the significant hurricanes and their evacuation types, it was noted that there was a correlation between the evacuation type and the amount of deaths and damages. The hurricanes with the most amount of deaths always occurred when there was either no evacuation or a voluntary evacuation announced. For the hurricanes with the most amount of damages, those hurricanes always had mandatory evacuations.

\evacuationTable

Lastly, looking at the hurricane paths, there are no apparent patterns that can be found through this data. Hurricane paths seem to be very random and no common tendencies. However, it is safe to say that the south eastern parts of the United States and areas near Cuba receive the brunt of most of the hurricanes indicating that those are some of the areas that should be well prepared for future hurricanes.

\section{Discussion}

One of the main takeaways from the results is that before the year 2000, there were a lot more deaths instead of damages and after the year 2000, there were a lot more damages in stead of deaths. This could be due to a number of reasons. For one, a lot of the earlier hurricanes had 0 for their damage number which usually meaning that the amount of damages was unknown. So even though some of the older hurricanes had damages recorded, it is unlikely that a majority of them did not suffer any damages at all. Thus the data before the year 2000 damage total could be inaccurate due to missing information.

Looking at evacuation types, many of the high death hurricanes had no evacuation or voluntary evacuations issued while most of the high damage hurricanes all had mandatory evacuations. This could be due to any number of reasons. One potential reason for the high number of deaths for voluntary evacuations is that it is not mandatory. Thus people think that it is still OK to stay in the area if they want to, not realizing how bad a hurricane can be. For Mandatory evacuations, the high number of damages could be because there is no one there to make sure stuff doesn't break. Since most people evacuated, the death count is a lot lower but then there is no way to protect buildings and belongings so a lot of destruction happened. This could be an indicator that people need to start thinking about how to safely equip them homes with hurricane proof materials in order to reduce the amount of damages during a dangerous hurricane. 

Given more time, it would have been nice to actually get user feedback on the animation to see if they understand the statement that was trying to be portrayed. User feedback could have helped in identifying what parts of the animation are good and what parts need to be emphasized more. Additionally, not having information about evacuation dates for all hurricanes and a data set readily available for deaths and damages was a big hindrance if making the visualization and given more time, there could have been more hurricanes displayed as significant hurricanes. Also, if there were information about the area of effect of each hurricane and a data set on regions in each state/country, it could be interesting to see which areas are effected the most by the hurricanes. 

\section{Conclusion}
\label{sec:conclusion}

This project aimed to portray a story of hurricane ignorance through an animation. This animation showed hurricane paths as well as the amount of deaths and damages accumulated from all the hurricanes. Additionally, significant hurricanes were showcased to help tell the story by displaying the type of evacuation issued during the hurricane. From the analysis of the animation, it was easy to see that many of the hurricane deaths happened earlier on before the year 2000 whereas most of the damages were done after the year 2000. This could be due to a majority of factors however, it is assumed that evacuation plans are becoming better thus why there are less deaths in the more recent hurricanes. The government should also consider issuing more mandatory evacuations because many people tend to not listen to the evacuation notices and end up suffering because of it.

In the future, gathering information about area of effect of each hurricane would benefit in figuring out what areas are damaged the most by the hurricanes. Additionally, gathering more information about each hurricane's evacuation type and gathering more hurricanes could help benefit this animation's story. Finally, polishing the animation so that the user could potentially zoom in on areas in the animation and projecting the map in each line graphs would also benefit in understanding where the hurricanes are affecting.