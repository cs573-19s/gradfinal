\documentclass[journal]{vgtc}

% TVCG Packages
\usepackage{mathptmx}
\usepackage{graphicx}
\usepackage{times}
\usepackage{multirow}
\usepackage{booktabs}
\usepackage{amsmath}
\usepackage{caption}
\usepackage{subcaption}
\usepackage[utf8]{inputenc}


\input{packages.tex}
\newcommand\multipleViews{

\begin{figure}
  \centering
  \includegraphics[width=\columnwidth]{img/brushLineGraphs}
  \caption{
    Filtered multiple views of individual hurricanes in a specified area from the brush in the animation.
  }
  \label{fig:protein-validate}
\end{figure}

}

\newcommand\brush{

\begin{figure}
  \centering
  \includegraphics[width=\columnwidth]{img/brush}
  \caption{
    Brush filter in the animation.
  }
  \label{fig:protein-validate}
\end{figure}

}
\newcommand\deathsDamagesTable{

\begin{table}[]
\centering

\begin{tabular}{r c c c}

\multicolumn{1}{l|}{\textbf{Measures}} & \textbf{Before 2000} & \textbf{After 2000} & \textbf{Total}\\
\hline
\multicolumn{1}{l|}{Deaths}  & 67,883  & 5,765 & 73,648  \\ 
\multicolumn{1}{l|}{Damages}  & \$78,399M  & \$317,965M & \$396,364M  \\

\end{tabular}

\caption{Number of deaths and damages before and after the year 2000}
\label{tab:results}
\end{table}

}

\newcommand\evacuationTable{

\begin{table}[]
\centering

\begin{tabular}{r c c c}

\multicolumn{1}{l|}{\textbf{Hurricane}} & \textbf{Evacuation Type} & \textbf{Deaths} & \textbf{Damages}\\
\hline
\multicolumn{1}{l|}{Galveston 1900}  & None  & 12000 & \$30M  \\ 
\multicolumn{1}{l|}{Flora 1963}  & Voluntary  & 8000 & Unknown  \\ 
\multicolumn{1}{l|}{FiFi 1974}  & Voluntary  & 8000 & Unknown  \\ 
\multicolumn{1}{l|}{Mitch 1998}  & Voluntary  & 9086 & \$40M  \\ 
\multicolumn{1}{l|}{Jeanne 2004}  & Voluntary  & 3000 & \$7,660M  \\ 
\multicolumn{1}{l|}{Andrew 1992}  & Mandatory  & 26 & \$26,500M  \\ 
\multicolumn{1}{l|}{Katrina 2005}  & Mandatory  & 1200 & \$108,000M  \\ 
\multicolumn{1}{l|}{Wilma 2005}  & Mandatory  & 23 & \$21,007M  \\ 
\multicolumn{1}{l|}{Ike 2008}  & Mandatory  & 103 & \$29,520M  \\ 
\multicolumn{1}{l|}{Sandy 2012}  & Mandatory  & 147 & \$75,000M  \\ 

\end{tabular}

\caption{Comparing significant hurricanes based on evacuation type}
\label{tab:results}
\end{table}

}

\usepackage[bookmarks,backref=true,linkcolor=black]{hyperref} %,colorlinks
\hypersetup{
  pdfauthor = {},
  pdftitle = {},
  pdfsubject = {},
  pdfkeywords = {},
  colorlinks=true,
  linkcolor= black,
  citecolor= black,
  pageanchor=true,
  urlcolor = blue,
  plainpages = false,
  linktocpage
}

%% If you are submitting a paper to a conference for review with a double
%% blind reviewing process, please replace the value ``0'' below with your
%% OnlineID. Otherwise, you may safely leave it at ``0''.
\onlineid{0}

%% declare the category of your paper, only shown in review mode
\vgtccategory{Exploration}

\vgtcinsertpkg


\title{Ignorance of Hurricanes}

\author{Jason Abel}
\authorfooter{
\item
 Jason Abel is with Worcester Polytechnic Institute. E-mail: jabel@wpi.edu

}

%other entries to be set up for journal
% \shortauthortitle{Author \MakeLowercase{\textit{et al.}}: Title}



%% Abstract section.
\abstract{
The objective of this paper is to inform people of the dangers of hurricanes. Over the years many people have died to hurricanes and there have been many damages caused by hurricanes. This paper takes a look at hurricane paths, death count, damage count, and evacuation types of larger scale hurricanes in the Atlantic ocean. It was seen from the information gathered that people may underestimate hurricanes and may not prepare for hurricanes well due to the large amounts of death and damages accumulated over time. It is estimated that if we took all the money accumulated in damages, we could produce  ___ hurricane proof rooms.

This paper aims to get people to understand that hurricanes are unpredictable and that people should not waste time in evacuating when told to do so and to prepare ahead of time. This paper will talk a little bit about the background of hurricane related information and then the methodology of how an animation was created in order to tell a story about hurricanes. Finally, the results found will be stated and discussed along with a final conclusion made about the project.
} 

\keywords{Hurricanes, Visualization, Animation, Story, Information}

%% ACM Computing Classification System (CCS). 
%% See <http://www.acm.org/class/1998/> for details.
%% The ``\CCScat'' command takes four arguments.

% \CCScatlist{ % not used in journal version
%  \CCScat{K.6.1}{Management of Computing and Information Systems}%
% {Project and People Management}{Life Cycle};
%  \CCScat{K.7.m}{The Computing Profession}{Miscellaneous}{Ethics}
% }

%% Uncomment below to include a teaser figure.
% Teaser should be jnds for -pcp versus +pcp (distinguishable; not distinguishable; target; not distinguishable; distinguishable)
\teaser{
  \centering
  \includegraphics[width=\textwidth]{img/teaser}
  \caption{Hurricane animation currently displaying Hurricane Andrew}
  \label{fig:teaser}
}

%% Uncomment below to disable the manuscript note
%\renewcommand{\manuscriptnotetxt}{}

%% Copyright space is enabled by default as required by guidelines.
%% It is disabled by the 'review' option or via the following command:
% \nocopyrightspace

%%%%%%%%%%%%%%%%%%%%%%%%%%%%%%%%%%%%%%%%%%%%%%%%%%%%%%%%%%%%%%%%
%%%%%%%%%%%%%%%%%%%%%% START OF THE PAPER %%%%%%%%%%%%%%%%%%%%%%
%%%%%%%%%%%%%%%%%%%%%%%%%%%%%%%%%%%%%%%%%%%%%%%%%%%%%%%%%%%%%%%%%

\begin{document}

%% The ``\maketitle'' command must be the first command after the
%% ``\begin{document}'' command. It prepares and prints the title block.

%% the only exception to this rule is the \firstsection command
\firstsection{Introduction}

\maketitle

%% \section{Introduction} %for journal use above \firstsection{..} instead

% Introduction section is automatically added
As a major public transportation, airline market has been ushering a booming development. On a global scale, a continuous world-wide growth of air traffic could be observed, and according to several market researches, the growth is expected to maintain positive rates up to 2030.

However, there are many factors that affect the performance of the commercial aviation system, which can lead to annoying results to their passengers sometimes. Given the uncertain factors of the whole aviation system, passengers usually have to plan their travel many days or even months before the departure date. Meanwhile, in order to decrease the trip costs, avoid the rush traffic hours, and then obtain a relaxed travel experience, travelers also hope to gain as more detailed information as possible.

Converting the traditional numeric information into a more vivid visualization form, could help the viewers gain their desired information efficiently and easily. So we intend to build a map based interactive consulting visualization, which combines two date sets coming from the US Department of Transportation Bureau of Transportation Statistics. We hope this application can reveal some potential patterns under the flight records and display them to the viewers.

We apply D3.js library to build the whole data visualization including one map view which is used to depict the airports and the airlines. While users move the mouse over an area belonged to a specific state, the area will highlight of which the color change and the name of that state will pop up. Once users click the state, it will filter out other states, that only displays the airports sited inside the chosen state. For sure, we also provide a button placed at the top middle of the web page, to reset the map view to its original status. Correspondingly, the right-side bar chart will change once users click a specific state. The bar chart we use here is to display a comparison of fight delay period among at most six airline companies. This bar chart is a variety of the plain bar, which is divided horizontally into upper and lower part, each of them present the degree of departure and arrival delay respectively. Initially, this stack bar chart will display average delay time across all the flight data. Of course, to make the bar chart more practical, we also append the numeric value of the flight delay time while users move the mouse over a specific part of the bar chart. Meanwhile, the word cloud view will change as a reaction of the click action on the stack bar chart, the key words it shows every time will change correspond with the content of passengers’ tweets. Most of the words are the reflection of the feel of the passengers, either is positive or negative.



%\S\ref{sec:conclusion}.

\testfigure

%\testtable

\section{Background}
As the time goes by, the use of coordinated multiple views has been changing and expanding a lot, in addition it also becomes part of larger sense making environments where the techniques are being used to analyze large datasets, integrate alternate viewpoints, and generate nuggets of information.\cite{roberts2007state} Nowadays, D3.js library is one of the most popular tools to implement the coordinated multiple views and then analysis large data set. It is worth and quite practical to apply these ideas and tools while building the final project. Meanwhile, data visualization is not just a way that simply transforms the data into several tables or charts, instead, it also involves pre-processing on data such as clearing, filtering, mapping or other aggregation operation. The process that chose appropriate visualization pattern with the dataset is challenging, but on the other hand, a good vis always provides its audiences an intuitive and logical experience. Utilizing all the handy techniques we have to develop an extension of the previous project, is a good study path for our future work.

There are three important parts: designing the whole visualization, fixing the big data problem (our original dataset contains 300M+ rows), achieving the interactions between visualizations with the help of d3 and react instead of using dispatch.

\section{Method}

Basically we used D3.js to build each data visualization including US Map, Stacked Bar Chart, Word Cloud. 

\subsection{Interaction}
People usually used dispatch which is a tool to create interaction between data visualization views to create interactions, we decide to use React.js to do the same thing instead of dispatch.

React is a JavaScript library for building user interfaces, the core idea behind it is integrating separate parts into a whole thing (class) and controlling each part's state. For example, we can create a simple spinning button by combining a button element, a spinning figure, and a state variable to control spinning or not. We embed the three things into a class SpinningButton, we set the spinning state with false as default, when user click the button, setting the state with true, showing the spinning figure and starting up a timer function as the same time, when time running out, set the state with false back, hidden the spinning figure. 

As you can see, we can use this kind of idea to control our visualization information and transport the information by react states, the figure~\ref{fig:stateEx} shows our state. 
\stateEx



\subsection{Big Data Problem}

As we mentioned before, our original dataset contains 300M+ rows, the data size is above 600MB, it is definitely not a good idea to store the data along with the website. Instead of uploading the dataset, we create a MongoDB (a NoSQL database) database to store our data. There are two reasons we choose MongoDB, the first reason is it is NoSQL data structure as JSON format, it's convenient to I/O data. It has a good compatible with Amazon Website Service which is also good for transporting data. Of course, MongoDB is also good for aggregation operation which help us to retrieve different kind of query combinations. 

With the help of MongoDB and AWS, we don't need to uploading the whole large dataset which could make the website run slowly. On the other hand, we can keep the whole dataset without any pruning and modification which could lead information loss.

We can see in figure~\ref{fig:mongoEx}, our flights collection is up to 1.2G which is super large.
\mongodbEx

\section{D3 Transition}
When the data changes in a d3 views, what is the interesting thing we may notice? That is transition. It's boring if a data visualization just change their view by refreshing the whole view, that is deleting the whole dataset, and replacing with new dataset. D3 provides a really good way to deal with this problem, it's called, enter, update and exit as shown in figure~\ref{fig:transition}. It's super useful and interesting part of D3.js and also it's hard to understand, as lease we spent a lot of time to understand these kind of concepts. Generally, the processes are, supposing our view has been implemented. Now, we need to update the view by the new dataset. First, the D3 collects the previous dataset and new coming dataset, and secondly, the D3 finds the common part and different part (we could or we better specify which data field we are going to compare ), and lastly, the D3 update the different part and remove the useless data. From these procedures, we could efficiently and perfectly control the transitions. 
\transition
\section{Results}
With the help of D3.js, React.js, MongoDB and Amazon Website Service, we successfully create our whole interactive data visualization. Our visualization shows the average airline departure delay time, the average airline arrival delay time and Tweeter user's sentiment. User could click any state to show the information corresponding the all airports in this specific state. User also could click the stack bar chart to show each specific airline's corresponding information. 

\section{Discussion}
It gives an idea that how to depict a visualization of adjacency relations in hierarchical data, especially with a huge dataset. \cite{holten2006hierarchical}

They presented a visual analysis of Twitter time-series, which combines sentiment and stream analysis with geoand time-based interactive visualizations for the exploration of real-world Twitter data streams. \cite{hao2011visual}

It introduced a novel visualization called NodeTrix.\cite{henry2007nodetrix} 

It restyles many useful and powerful d3 data visualizations.\cite{harper2014deconstructing}




\subsection{Conclusion}
\label{sec:conclusion}
This paper is mainly trying to find a way to create a interactive visualization by combining React.js and D3.js. The visualization is also helpful for people who want to know what is the average delay of each airlines and what is the people's attitude by presenting Tweeter's sentiment information. 

It also find a way to deal with big data problem when the dataset it's too large to upload to website. 

Moreover, it gives a good example to use D3 transition for improving the visualization's quality.

At the end, we find that combining React and D3 is a good way to develop interactive visualization, it's efficient, easy to control and extensible. Although our current visualization's interaction is not fast, that is because the transporting problem between the AWS and MongoDB, and the dataset is too large to do complicate query operation like finding all given airline in some airports. That is nothing to do with the React and D3.

In the future, we could add more views to allow user to discover more information. For example, we could add a timeline to find the delay information according the time range. We also could add a functionality that user could click any two of airports to see the corresponding information between the two airports, or click any two of us states to show corresponding information.




%% if specified like this the section will be committed in review mode
\acknowledgments{
The Author would like to acknowledge Lane Harrison for his insight in helping come up with ideas on how to build the visualization and the guidance provided throughout the project. 
}

\bibliographystyle{abbrv}
\bibliography{paper.bib}


\end{document}